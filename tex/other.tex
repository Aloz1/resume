\newcommand*\showaccomplishments{
    \section{Extracurricular}
    \cvitem{2020 - Present}{Assistant Scout Leader\hfill{\usefontofkomafont{commentfont} 8th Knox Scouts}}
    \cvitem{2015 - Present}{Various CTFs\hfill{\usefontofkomafont{commentfont} SquareCTF, Over The Wire, Cybar, PicoCTF, CSAW}}
    \cvitem{2017, 2018}{Cyber Security Challenge Australia (CySCA)}
    \cvitem{2012}{YMCA UNO-Y Youth Leadership Program}
    \cvitem{2011, 2012}{YMCA Youth Parliament Victoria}
}

\newcommand*\showmemberships{
    \section{Clubs and associations}
    \cvitem{2015 - Present}{Engineers Australia}
    \cvitem{2020 - Present}{Scouts Victoria - Leader}
    \cvitem{2015 - 2020}{Scouts Victoria - Rovers}
    \cvitem{2015 - 2019}{Swinburne Cyber Security Club}
}

\newcommand*\showpasttimes{
    \section{Hobbies and Sports}
    % Looks ugly, need to make it look a bit better.
    \cvitem{}{Tinkering with electronics and computers}
    \cvitem{}{Camping and 4x4ing}
    \cvitem{}{Skiing\hfill{\usefontofkomafont{commentfont} Cross country, Down hill}}
}

\newcommand*\showprojects{
    \section{Projects}
    \cventry{2022 - Present}{VSWR Meter}{}{}{}{
        A hardware tool used for checking impedance matching of RF devices.
        \vskip .5em
        \begin{compactitem}
            \item Will use as a tool to help further my understanding pf RF equipment
            \item Using KiCAD for schematic capture and PCB draughting.
            \item Warly stage of development, not yet complete.
        \end{compactitem}
    }
    \cventry{2021}{VHDL Sinusoidal CORDIC}{}{}{}{
        Simple sinusoidal CORDIC function, with the ultimste goal of utilising it
        to eventually build a Vector Network Analyser.
        \vskip .5em
        \begin{compactitem}
            \item Used to help familiarise myself with MicroSemi tooling.
            \item Verified using cocotb and vunit.
        \end{compactitem}
    }
    \cventry{2019}{Computer vision based bus passenger tracker}{}{}{}{
        Prototype software for tracking passengers as they enter and leave a bus.
        \vskip .5em
        \begin{compactitem}
            \item Built on a Raspberry Pi + Intel Neural Compute Stick platform.
            \item Configuration app with live video streaming via GStreamer RTSP.
            \item Master/Slave devices with HTTP communication to the cloud.
            \item Written in C \& C++ using GCC and CMake.
        \end{compactitem}
    }\vskip .5em
    \cventry{2019}{IoT Vehicle tracking system}{}{}{\hfill\httpslink{github.com/Aloz1/iot-report}}{
        A proof of concept IoT vehicle tracking system with wireless BLE nodes and cloud data storage.
        \vskip .5em
        \begin{compactitem}
            \item Includes an ESP32 based GPS node and an Arduino based IMU node with BLE GATT communications to an edge device (raspberry pi).
            \item Edge device utilises Redis data caching, for intermittent internet connectivity.
            \item Communications between edge and cloud is via MQTT, with data stored in DynamoDB.
            \item Firmware for nodes written in C++, edge services written in Python, web client uses a combination of Python, javascript and HTML, the report is written in \LaTeX.
        \end{compactitem}
    }\vskip .5em
    \cventry{2018}{Zynq Space Invaders}{}{}{\hfill\httpslink{github.com/Aloz1/Zybo-SpaceInvaders}}{
        University project demonstrating how a Zynq FPGA may be used with FreeRTOS and custom logic
        to build a small space invaders clone.
        \vskip .5em
        \begin{compactitem}
            \item Custom FPGA logic for direct communications with a PlayStation controller.
            \item Designed such that each space invader operates in their own thread/task to demonstrate concurrency and task scheduling.
            \item FPGA logic developed with VHDL and Vivado block designer.
            \item Space invaders game written in C++ and is designed to run on top of FreeRTOS.
        \end{compactitem}
    }\vskip .5em
}
